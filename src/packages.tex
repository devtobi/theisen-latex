%%% === BEGINN DER KONFIGURATION === %%%

% Stichwortverzeichnis
\usepackage{makeidx}
\makeindex
\renewcommand{\indexname}{Stichwortverzeichnis}

% PDF-Inhalt such- und kopierbar machen
\usepackage[resetfonts]{cmap}

% Anpassungen für deutsche Sprache + englische Sprache
\usepackage[utf8]{inputenc}
\usepackage[T1]{fontenc}
\usepackage[ngerman,english]{babel}

% Mehr Farben zulassen
\usepackage[table,dvipsnames]{xcolor}

% Grafiken
\usepackage{graphicx}
\graphicspath{{images/}}
\usepackage{float}

% Textschriftart ändern
\usepackage{lmodern}

% Verbessertes Fontrendering
\usepackage[final]{microtype}

% Seitenränder festlegen
\usepackage[\papersize,left=\leftpapermargin,right=\rightpapermargin,top=\toppapermargin,bottom=\bottompapermargin]{geometry}

% 1,5 Zeilenabstand konfigurieren
\usepackage[onehalfspacing]{setspace}

% Zeileneinschub mit Abstand ersetzen
\usepackage{parskip}

% Fußnoten konfigurieren
\usepackage[perpage, flushmargin, hang]{footmisc}
\setlength{\footnotemargin}{4mm}

% Namensnennungen kursiv schreiben
\usepackage{xpatch}
\xpatchcmd{\citeauthor}{\begingroup}{\begingroup\em}{}{}

% Inhaltsverzeichnis Aussehen konfigurieren und ins Inhaltsverzeichnis mit aufnehmen
\usepackage{tocbibind}
\usepackage{tocloft}
\renewcommand{\cftsecleader}{\cftdotfill{\cftdotsep}}
\setlength{\cftsubsecindent}{\idxsubsecindent}
\setlength{\cftsubsubsecindent}{\idxsubsubsecindent}
\setlength{\cftbeforesecskip}{3pt}
\setlength{\cftfigindent}{0pt}
\setlength{\cfttabindent}{0pt}


% Schönere Seitenangaben
\usepackage{fancyhdr}
\fancyhf{}
\fancyhead[EL,OR]{\thepage}
\fancyhead[ER]{\chaptername~\thechapter}
\fancyhead[LO]{\nouppercase{\leftmark}}

% Mathematik-Pakete
\usepackage{amsmath,amsthm,amssymb}
\usepackage{siunitx}
\setlength{\mathindent}{1cm}

% Automatisches Setzen von Anführungszeichen (\enquote, \foreignquote, \blockquote)
\usepackage[autostyle=true,german=quotes,babel]{csquotes}

% Einbinden von PDFs
\usepackage{pdfpages}

% Erlaubt die Erstellung von zusammengesetzten Grafiken
\usepackage{subcaption}

% Leere Anfangs- und Endzeilen vermeiden
\usepackage[defaultlines=4,all]{nowidow}

% Bessere Aufzählungen
\usepackage{paralist}

% Bessere Tabellen
\usepackage{booktabs}

% Euro-Zeichen
\usepackage[official]{eurosym}

% Abkürzungen
\usepackage[footnote, printonlyused, smaller]{acronym}

%%% === LISTINGS === %%%
\usepackage{listings}
\usepackage{beramono}
\renewcommand{\lstlistingname}{Quellcode}

% CSS als Sprache definieren
\lstdefinelanguage{CSS}{
  keywords={color,background-image:,margin,padding,font,weight,display,position,top,left,right,bottom,list,style,border,size,white,space,min,width, transition:, transform:, transition-property, transition-duration, transition-timing-function},	
  sensitive=true,
  morecomment=[l]{//},
  morecomment=[s]{/*}{*/},
  morestring=[b]',
  morestring=[b]",
  alsoletter={:},
  alsodigit={-}
}

% JavaScript als Sprache definieren
\lstdefinelanguage{JavaScript}{
  keywords={break, case, catch, continue, debugger, default, delete, do, else, finally, for, function, if, in, instanceof, new, return, null, true, false, switch, this, throw, try, typeof, var, void, while, with},
  ndkeywords={class, export, boolean, throw, implements, import, this},
  sensitive=true,
  morecomment=[l]{//},
  morecomment=[s]{/*}{*/},
  morestring=[b]',
  morestring=[b]"
}

% HTML5 als Sprache definieren
\lstdefinelanguage{HTML5}{
  language=html,
  sensitive=true,	
  alsoletter={<>=-},	
  morecomment=[s]{<!-}{-->},
  tag=[s],
  otherkeywords={
  % General
  >,
  % Standard tags
	<!DOCTYPE,
  </html, <html, <head, <title, </title, <style, </style, <link, </head, <meta, />,
	% body
	</body, <body,
	% Divs
	</div, <div, </div>,
	% headings
	</h1, <h1, </h1>,
	</h2, <h2, </h2>, 
	</h3, <h3, </h3>, 
	</h4, <h4, </h4>, 
	</h5, <h5, </h5>,
	</h6, <h6, </h6>,  
	% Paragraphs
	</p, <p, </p>,
	% scripts
	</script, <script,
  % More tags...
  <canvas, /canvas>, <svg, <rect, <animateTransform, </rect>, </svg>, <video, <source, <iframe, </iframe>, </video>, <image, </image>, <header, </header, <article, </article
  },
  ndkeywords={
  % General
  =,
  % HTML attributes
  charset=, src=, id=, width=, height=, style=, type=, rel=, href=,
  % SVG attributes
  fill=, attributeName=, begin=, dur=, from=, to=, poster=, controls=, x=, y=, repeatCount=, xlink:href=,
  % properties
  margin:, padding:, background-image:, border:, top:, left:, position:, width:, height:, margin-top:, margin-bottom:, font-size:, line-height:,
	% CSS3 properties
  transform:, -moz-transform:, -webkit-transform:,
  animation:, -webkit-animation:,
  transition:,  transition-duration:, transition-property:, transition-timing-function:,
  }
}

% Kombination aus HTML,CSS,JS als Style definieren
\lstdefinestyle{htmlcssjs} {%
  % Code
  language=HTML5,
  alsolanguage=JavaScript,
  alsodigit={.:;},	
}

\lstset{
	breaklines=true,
	breakatwhitespace=false,
	captionpos=b,
	frame=shadowbox,
	columns=flexible,
	keepspaces=false,
	numbers=left,
	showspaces=false,
	showstringspaces=false,
	showtabs=false,
	identifierstyle=\color{black},
	ndkeywordstyle=\color{MidnightBlue}\bfseries,
	basicstyle=\footnotesize\ttfamily,
	numberstyle=\color{Gray},
	keywordstyle=\bfseries\color{RoyalBlue},
	commentstyle=\color{ForestGreen},
	stringstyle=\color{Orange},
	xleftmargin={0.75cm},
	tabsize=2,
	extendedchars=true,
	inputencoding=utf8,
	resetmargins=true,
	rulesepcolor=\color{lightgray},
	literate=
  {á}{{\'a}}1 {é}{{\'e}}1 {í}{{\'i}}1 {ó}{{\'o}}1 {ú}{{\'u}}1
  {Á}{{\'A}}1 {É}{{\'E}}1 {Í}{{\'I}}1 {Ó}{{\'O}}1 {Ú}{{\'U}}1
  {à}{{\`a}}1 {è}{{\`e}}1 {ì}{{\`i}}1 {ò}{{\`o}}1 {ù}{{\`u}}1
  {À}{{\`A}}1 {È}{{\'E}}1 {Ì}{{\`I}}1 {Ò}{{\`O}}1 {Ù}{{\`U}}1
  {ä}{{\"a}}1 {ë}{{\"e}}1 {ï}{{\"i}}1 {ö}{{\"o}}1 {ü}{{\"u}}1
  {Ä}{{\"A}}1 {Ë}{{\"E}}1 {Ï}{{\"I}}1 {Ö}{{\"O}}1 {Ü}{{\"U}}1
  {â}{{\^a}}1 {ê}{{\^e}}1 {î}{{\^i}}1 {ô}{{\^o}}1 {û}{{\^u}}1
  {Â}{{\^A}}1 {Ê}{{\^E}}1 {Î}{{\^I}}1 {Ô}{{\^O}}1 {Û}{{\^U}}1
  {Ã}{{\~A}}1 {ã}{{\~a}}1 {Õ}{{\~O}}1 {õ}{{\~o}}1
  {œ}{{\oe}}1 {Œ}{{\OE}}1 {æ}{{\ae}}1 {Æ}{{\AE}}1 {ß}{{\ss}}1
  {ű}{{\H{u}}}1 {Ű}{{\H{U}}}1 {ő}{{\H{o}}}1 {Ő}{{\H{O}}}1
  {ç}{{\c c}}1 {Ç}{{\c C}}1 {ø}{{\o}}1 {å}{{\r a}}1 {Å}{{\r A}}1
  {€}{{\euro}}1 {£}{{\pounds}}1 {«}{{\guillemotleft}}1
  {»}{{\guillemotright}}1 {ñ}{{\~n}}1 {Ñ}{{\~N}}1 {¿}{{?`}}1
}
%%% === ENDE LISTINGS === %%%

% Beschriftung von Bildern/Tabellen/Listings usw.
\usepackage[
	format=plain,
	margin=0.5cm,
	labelformat=simple,
	labelfont={bf},
	font=footnotesize
]{caption}
\usepackage{chngcntr}
\counterwithin{figure}{section}
\counterwithin{table}{section}
\counterwithin{equation}{section}
\AtBeginDocument{\counterwithin{lstlisting}{section}}

% Literaturverzeichnis & Harvard-Stil einrichten
% Chicago-Stil
%\usepackage[
%authordate,
%backend=biber,
%bibencoding=inputenc,
%language=ngerman,
%footmarkoff,
%alwaysrange=true,
%natbib,
%]{biblatex-chicago}
%\renewcommand{\postnotedelim}{%
%  \iffieldpages{postnote}
%    {\addcomma\space S.}
%    {\addspace}}

% Harvard-Stil
\usepackage[style=authoryear,backend=biber]{biblatex}
\renewcommand*{\finentrypunct}{}
\setlength\bibitemsep{0pt}
\setlength\bibinitsep{10pt}
\renewcommand{\mkbibnamelast}[1]{\mkbibemph{#1}}
\renewcommand{\mkbibnamefirst}[1]{\mkbibemph{#1}}
\renewcommand*{\nameyeardelim}{\addcomma\space}
\renewcommand{\UrlFont}{\footnotesize\ttfamily}

\addbibresource{bibliography.bib}

%Verlinkungen und PDF Daten initialisieren
\usepackage{hyperref}
\hypersetup{
	pdftitle = {\thesistitle},
	pdfauthor = {\thesisauthor},
	pdfcreator = {pdflatex},
	pdfkeywords={\thesistype, \thesistitle},
	pdfsubject={\thesistype},
	pdfproducer = {LaTeX with hyperref},
	colorlinks = false,
	linktoc = all,
	linkbordercolor = {0 0 0.5},
}
\usepackage[all]{hypcap}

%%% === ENDE DER KONFIGURATION === %%%