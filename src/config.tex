%% -- Allgemeine Konfiguration -- %%

% Textgröße
\newcommand{\maintextsize}{12pt}

% Papierart
\newcommand{\papersize}{a4paper}

% Ränder
\newcommand{\leftpapermargin}{4cm}
\newcommand{\rightpapermargin}{2cm}
\newcommand{\toppapermargin}{3.8cm}
\newcommand{\bottompapermargin}{2cm}

%% -- Deckblattkonfiguration -- %%

% Rand
\newcommand{\titlemargin}{2cm}
\newcommand{\lefttitlemargin}{\titlemargin}

% Name der Universität/Hochschule
\newcommand{\universityname}{Universitätsname}

% Name der Fakultät
\newcommand{\facultyname}{Fakultätsname}

% Dateiname des Logos
\newcommand{\logofilename}{logo}

% Arbeitstyp
\newcommand{\thesistype}{Bachelorarbeit}	

% Titel der Arbeit
\newcommand{\thesistitle}{Titel der Arbeit}
\newcommand{\thesistitleenglish}{Accessibility in native web applications}

% Untertitel der Arbeit
\newcommand{\thesissubtitle}{Untertitel der Arbeit}
\newcommand{\thesissubtitleenglish}{Analysis of implementation and evaluation options based on WCAG success criteria}

% Zielabschluss
\newcommand{\studygoal}{Bachelor of Science}

% Autor der Arbeit
\newcommand{\thesisauthor}{Autor}

% Studiengang
\newcommand{\studytype}{Studiengang}

% Matrikelnummer
\newcommand{\studentnumber}{Matrikelnummer}

% Semesternummer
\newcommand{\semesternumber}{Fachsemester}

% Mail
\newcommand{\mail}{.edu-Mail}

% Abgabetermin
\newcommand{\submissiondate}{Abgabetermin}

% Semester
\newcommand{\semester}{WS 2020}

% Betreuer (mit akademischen Titeln)
\newcommand{\supervisor}{Betreuer}


%% -- Sperrvermerk - Konfiguration -- %%

% Firmenname
\newcommand{\companyname}{Unternehmen XY}

% Typ der Arbeit auf Englisch
\newcommand{\thesistypeenglish}{bachelor thesis}

%% -- Inhaltsverzeichnis - Konfiguration -- %%

% Einzug Subsection
\newcommand{\idxsubsecindent}{0.25cm}

% Einzug Subsubsection
\newcommand{\idxsubsubsecindent}{0.6cm}


%% -- Vorwort - Konfiguration -- %%

% Ändern Sie den Namen des Abschnitts auf 'Vorbemerkung' wenn Sie sich sehr kurz halten möchten.
\newcommand{\preamblename}{Vorwort}

% Verfassungsort
\newcommand{\preamblelocation}{München}

% Verfassungsdatum
\newcommand{\preambledate}{im Winter 2020}

%% -- Nachwort - Konfiguration -- %%

% Verfassungsort
\newcommand{\epiloguelocation}{München}

% Verfassungsdatum
\newcommand{\epiloguedate}{im Sommer 2021}
