\section{Einleitung}\label{sec:einleitung}

% \section, \subsection, \subsubsection : Kapitel der Arbeit
% \label: Setzen von Labels: sec, subsec, subsubsec, tab, fig
% \autoref : Referenz auf Elemente (Tabellen, Kapitel usw.)
% \pageref : Referenz auf Seiten
% \ac und \acp: Referenz auf ein angelegtes Acronym
% \glqq text\grqq{}: Anführungszeichen
% \parencite: Zitieren einer Quelle mit optionaler Seitenangabe
% \citeauthor: Referenz auf Autor einer Quelle
% \citeyear: Referenz auf Jahr einer Quelle
% \psq: Seitenangabe folgend z.B. wie in \parencite[vgl.][1\psq]{quelle}
% \hyperfootnote : Fußnote mit Link
% \paragraph : Unterzeilung innerhalb Text ohne neues Kapitel z.B. \paragraph{Test}\mbox{}
% \textelp : Auslassungszeichen für direkte Zitate
% \footnote : Manuelle Fußnote
% compactitem : Aufzählung

Das ist das einleitende Kapitel\index{Kapitel} dieser Arbeit.\autocite[347--356]{einstein}

Das ist eine Referenz auf \autoref{fig:logo_anhang}, eine Abbildung aus dem Anhang.