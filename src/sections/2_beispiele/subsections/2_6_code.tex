\subsection{Source-Code}\label{subsec:code}
	
\begin{lstlisting}[style=htmlcssjs,float=h,caption={JavaScript Hello World Beispiel},label=lst:hello]
var x = "Hello World" //Das ist ein Kommentar
console.log(x.toUppercase());
/* Auch Zeilenkommentare
sind möglich! */
\end{lstlisting}
	
\autoref{lst:hello} ist ein einfaches Codebeispiel.

\begin{lstlisting}[style=htmlcssjs,float=h,caption={Einfaches HTML Beispiel},label=lst:html]
<!DOCTYPE html>
<html>
	<body class="Hallo">
		<h1>My First Heading</h1>
		<p>My first paragraph.</p>
	</body>
</html>
\end{lstlisting}

\includewebcode{Das ist ein Beispiel}{test.html}

Man kann auch Referenzen (z.b auf \autoref{lst:test.html}) automatisch generieren.