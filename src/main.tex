% Import der Config-Variablen
%% -- Allgemeine Konfiguration -- %%

% Textgröße
\newcommand{\maintextsize}{12pt}

% Papierart
\newcommand{\papersize}{a4paper}

% Ränder
\newcommand{\leftpapermargin}{4cm}
\newcommand{\rightpapermargin}{2cm}
\newcommand{\toppapermargin}{3.8cm}
\newcommand{\bottompapermargin}{2cm}

%% -- Deckblattkonfiguration -- %%

% Rand
\newcommand{\titlemargin}{2cm}

% Name der Universität/Hochschule
\newcommand{\universityname}{Universitätsname}

% Name der Fakultät
\newcommand{\facultyname}{Fakultätsname}

% Dateiname des Logos
\newcommand{\logofilename}{logo}

% Arbeitstyp
\newcommand{\thesistype}{Bachelorarbeit}	

% Titel der Arbeit
\newcommand{\thesistitle}{Titel}

% Zielabschluss
\newcommand{\studygoal}{Bachelor of Science}

% Autor der Arbeit
\newcommand{\thesisauthor}{Autor}

% Studiengang
\newcommand{\studytype}{Studiengang}

% Matrikelnummer
\newcommand{\studentnumber}{Matrikelnummer}

% Semesternummer
\newcommand{\semesternumber}{Fachsemester}

% Mail
\newcommand{\mail}{.edu-Mail}

% Abgabetermin
\newcommand{\submissiondate}{Abgabetermin}

% Semester
\newcommand{\semester}{WS 2020}

% Betreuer (mit akademischen Titeln)
\newcommand{\supervisor}{Betreuer}


%% -- Inhaltsverzeichnis - Konfiguration -- %%

% Einzug Subsection
\newcommand{\idxsubsecindent}{0.25cm}

% Einzug Subsubsection
\newcommand{\idxsubsubsecindent}{0.6cm}


%% -- Vorwort - Konfiguration -- %%
% Ändern Sie den Namen des Abschnitts auf 'Vorbemerkung' wenn Sie sich sehr kurz halten möchten.
\newcommand{\preamblename}{Vorwort}

% Verfassungsort
\newcommand{\preamblelocation}{München}

% Verfassungsdatum
\newcommand{\preambledate}{im Winter 2020}


% Import der Befehle
% Befehl um Textbox für den Druck zu generieren
\newcommand{\calibrationbox}[2]{% parameters: #1=width, #2=height
	\setlength{\unitlength}{1.0mm}%
	\begin{picture}(#1,#2)%
	\linethickness{0.05mm}%
	\put(0,0){\dashbox{0.2}(#1,#2)%
	{\parbox{#1mm}{%
	\centering\footnotesize 
	width $ = #1 \textrm{mm}$\\
	height $ = #2 \textrm{mm}$
	}}}\end{picture}
}

% Eigener Befehl für autoref + Seitenangabe
\newcommand{\autorefpage}[1]{\autoref{#1} (S. \pageref{#1})}

% Befehl für Quellenangaben direkt unter dem Bild (rechts)
\newcommand*{\sourceright}[1]{\par\raggedleft\footnotesize \textit{Quelle}:~#1}

% Befehl für Quellenangaben direkt unter dem Bild (zentriert)
\newcommand*{\sourcecenter}[1]{\par\centering\footnotesize \textit{Quelle}:~#1}

% Metadatenangaben
\author{\thesisauthor}
\title{\thesistitle}

\documentclass[\maintextsize, \papersize, fleqn, xcolor=dvipsnames]{scrartcl}

% PDF-Inhalt such- und kopierbar machen
\usepackage[resetfonts]{cmap}

% Anpassungen für deutsche Sprache
\usepackage[utf8]{inputenc}
\usepackage[T1]{fontenc}
\usepackage[ngerman]{babel}

% Mehr Farben zulassen
\usepackage[table,dvipsnames]{xcolor}

% Grafiken
\usepackage{graphicx}
\graphicspath{{images/}}
\usepackage{float}

% Textschriftart ändern
\usepackage{lmodern}

% Verbessertes Fontrendering
\usepackage[final]{microtype}

% Seitenränder festlegen
\usepackage[\papersize,left=\leftpapermargin,right=\rightpapermargin,top=\toppapermargin,bottom=\bottompapermargin]{geometry}

% 1,5 Zeilenabstand konfigurieren
\usepackage[onehalfspacing]{setspace}

% Zeileneinschub mit Abstand ersetzen
\usepackage{parskip}

% Fußnoten konfigurieren
\usepackage[perpage, flushmargin, hang]{footmisc}

% Inhaltsverzeichnis Aussehen konfigurieren und ins Inhaltsverzeichnis mit aufnehmen
\usepackage{tocbibind}
\usepackage{tocloft}
\renewcommand{\cftsecleader}{\cftdotfill{\cftdotsep}}
\setlength{\cftsubsecindent}{\idxsubsecindent}
\setlength{\cftsubsubsecindent}{\idxsubsubsecindent}
\setlength{\cftbeforesecskip}{3pt}

% Schönere Seitenangaben
\usepackage{fancyhdr}
\fancyhf{}
\fancyhead[EL,OR]{\thepage}
\fancyhead[ER]{\chaptername~\thechapter}
\fancyhead[LO]{\nouppercase{\leftmark}}

% Namensnennungen kursiv schreiben
\usepackage{xpatch}
\xpatchcmd{\citeauthor}{\begingroup}{\begingroup\em}{}{}

% Mathematik-Pakete
\usepackage{amsmath,amsthm,amssymb}
\setlength{\mathindent}{1cm}

% Automatisches Setzen von Anführungszeichen (\enquote, \foreignquote, \blockquote)
\usepackage[autostyle=true,german=quotes]{csquotes}

% Einbinden von PDFs
\usepackage{pdfpages}

% Beschriftung von Bildern/Tabellen/Listings usw.
\usepackage[
	format=plain,
	margin=0.5cm,
	labelformat=simple,
	labelfont={bf},
	font=footnotesize
]{caption}

% Erlaubt die Erstellung von zusammengesetzten Grafiken
\usepackage{subcaption}

% Leere Anfangs- und Endzeilen vermeiden
\usepackage[defaultlines=4,all]{nowidow}

% Bessere Aufzählungen
\usepackage{paralist}

% Bessere Tabellen
\usepackage{booktabs}

% Euro-Zeichen
\usepackage[official]{eurosym}

% Abkürzungen
\usepackage[footnote, printonlyused, smaller]{acronym}

%%% === LISTINGS === %%%
\usepackage{listings}
\usepackage{beramono}

% CSS als Sprache definieren
\lstdefinelanguage{CSS}{
  keywords={color,background-image:,margin,padding,font,weight,display,position,top,left,right,bottom,list,style,border,size,white,space,min,width, transition:, transform:, transition-property, transition-duration, transition-timing-function},	
  sensitive=true,
  morecomment=[l]{//},
  morecomment=[s]{/*}{*/},
  morestring=[b]',
  morestring=[b]",
  alsoletter={:},
  alsodigit={-}
}

% JavaScript als Sprache definieren
\lstdefinelanguage{JavaScript}{
  keywords={break, case, catch, continue, debugger, default, delete, do, else, finally, for, function, if, in, instanceof, new, return, null, true, false, switch, this, throw, try, typeof, var, void, while, with},
  ndkeywords={class, export, boolean, throw, implements, import, this},
  sensitive=true,
  morecomment=[l]{//},
  morecomment=[s]{/*}{*/},
  morestring=[b]',
  morestring=[b]"
}

% HTML5 als Sprache definieren
\lstdefinelanguage{HTML5}{
  language=html,
  sensitive=true,	
  alsoletter={<>=-},	
  morecomment=[s]{<!-}{-->},
  tag=[s],
  otherkeywords={
  % General
  >,
  % Standard tags
	<!DOCTYPE,
  </html, <html, <head, <title, </title, <style, </style, <link, </head, <meta, />,
	% body
	</body, <body,
	% Divs
	</div, <div, </div>,
	% headings
	</h1, <h1, </h1>,
	</h2, <h2, </h2>, 
	</h3, <h3, </h3>, 
	</h4, <h4, </h4>, 
	</h5, <h5, </h5>,
	</h6, <h6, </h6>,  
	% Paragraphs
	</p, <p, </p>,
	% scripts
	</script, <script,
  % More tags...
  <canvas, /canvas>, <svg, <rect, <animateTransform, </rect>, </svg>, <video, <source, <iframe, </iframe>, </video>, <image, </image>, <header, </header, <article, </article
  },
  ndkeywords={
  % General
  =,
  % HTML attributes
  charset=, src=, id=, width=, height=, style=, type=, rel=, href=,
  % SVG attributes
  fill=, attributeName=, begin=, dur=, from=, to=, poster=, controls=, x=, y=, repeatCount=, xlink:href=,
  % properties
  margin:, padding:, background-image:, border:, top:, left:, position:, width:, height:, margin-top:, margin-bottom:, font-size:, line-height:,
	% CSS3 properties
  transform:, -moz-transform:, -webkit-transform:,
  animation:, -webkit-animation:,
  transition:,  transition-duration:, transition-property:, transition-timing-function:,
  }
}

% Kombination aus HTML,CSS,JS als Style definieren
\lstdefinestyle{htmlcssjs} {%
  % Code
  language=HTML5,
  alsolanguage=JavaScript,
  alsodigit={.:;},	
}

\lstset{
	breaklines=true,
	breakatwhitespace=false,
	captionpos=b,
	frame=shadowbox,
	columns=flexible,
	keepspaces=false,
	numbers=left,
	showspaces=false,
	showstringspaces=false,
	showtabs=false,
	identifierstyle=\color{black},
	ndkeywordstyle=\color{MidnightBlue}\bfseries,
	basicstyle=\footnotesize\ttfamily,
	numberstyle=\color{Gray},
	keywordstyle=\bfseries\color{RoyalBlue},
	commentstyle=\color{ForestGreen},
	stringstyle=\color{Orange},
	xleftmargin={0.75cm},
	tabsize=2,
	extendedchars=true,
	inputencoding=utf8,
	resetmargins=true,
	rulesepcolor=\color{lightgray},
	literate=
  {á}{{\'a}}1 {é}{{\'e}}1 {í}{{\'i}}1 {ó}{{\'o}}1 {ú}{{\'u}}1
  {Á}{{\'A}}1 {É}{{\'E}}1 {Í}{{\'I}}1 {Ó}{{\'O}}1 {Ú}{{\'U}}1
  {à}{{\`a}}1 {è}{{\`e}}1 {ì}{{\`i}}1 {ò}{{\`o}}1 {ù}{{\`u}}1
  {À}{{\`A}}1 {È}{{\'E}}1 {Ì}{{\`I}}1 {Ò}{{\`O}}1 {Ù}{{\`U}}1
  {ä}{{\"a}}1 {ë}{{\"e}}1 {ï}{{\"i}}1 {ö}{{\"o}}1 {ü}{{\"u}}1
  {Ä}{{\"A}}1 {Ë}{{\"E}}1 {Ï}{{\"I}}1 {Ö}{{\"O}}1 {Ü}{{\"U}}1
  {â}{{\^a}}1 {ê}{{\^e}}1 {î}{{\^i}}1 {ô}{{\^o}}1 {û}{{\^u}}1
  {Â}{{\^A}}1 {Ê}{{\^E}}1 {Î}{{\^I}}1 {Ô}{{\^O}}1 {Û}{{\^U}}1
  {Ã}{{\~A}}1 {ã}{{\~a}}1 {Õ}{{\~O}}1 {õ}{{\~o}}1
  {œ}{{\oe}}1 {Œ}{{\OE}}1 {æ}{{\ae}}1 {Æ}{{\AE}}1 {ß}{{\ss}}1
  {ű}{{\H{u}}}1 {Ű}{{\H{U}}}1 {ő}{{\H{o}}}1 {Ő}{{\H{O}}}1
  {ç}{{\c c}}1 {Ç}{{\c C}}1 {ø}{{\o}}1 {å}{{\r a}}1 {Å}{{\r A}}1
  {€}{{\euro}}1 {£}{{\pounds}}1 {«}{{\guillemotleft}}1
  {»}{{\guillemotright}}1 {ñ}{{\~n}}1 {Ñ}{{\~N}}1 {¿}{{?`}}1
}

%%% === ENDE LISTINGS === %%%

% Verlinkungen und PDF Daten initialisieren
\usepackage{hyperref}
\hypersetup{
	pdftitle = {\thesistitle},
	pdfauthor = {\thesisauthor},
	pdfcreator = {pdflatex},
	pdfkeywords={\thesistype, \thesistitle},
	pdfsubject={\thesistype},
	pdfproducer = {LaTeX with hyperref},
	colorlinks = false,
	linktoc = all,
	linkbordercolor = {0 0 0.5},
}
\usepackage[all]{hypcap}

%%% === ENDE DER KONFIGURATION === %%%



%%% === BEGINN DES DOKUMENTS === %%%

\begin{document}

%Namen der autoref generierten Verweise ändern
\renewcommand{\sectionautorefname}{Kapitel}
\renewcommand{\subsectionautorefname}{Abschnitt}
\renewcommand{\subsubsectionautorefname}{Unterabschnitt}

%Seitennummerierung abschalten
\pagestyle{empty}
	
%Deckblatt einbinden
%Seitenränder abschalten
\newgeometry{left=\titlemargin,right=\titlemargin,top=\titlemargin,bottom=\titlemargin}

\begin{titlepage}

   \singlespacing

   %Inhalt zentrieren
   \begin{center} \large 
    
    \universityname
    \vspace*{0.5cm}
    
    \facultyname
    \vspace*{1.5cm}
    
	\includegraphics[scale=0.45]{titlepage/\logofilename}
    \vspace*{2cm} 

	\thesistype \ zum Thema
	\vspace*{0.5cm}
	
    {\huge\textbf{\thesistitle}}
    \vspace*{1.5cm}
    
    
  	Zur Erlangung des Grades \studygoal
  	\vspace*{1.5cm}

    \textbf{\thesisauthor}
    \vspace*{0.5cm}
    
    Studiengang: \studytype \\
    Matrikelnummer: \studentnumber \\
    Fachsemester: \semesternumber \\
    E-Mail: \mail \\
    \vspace*{2.0cm}

    Abgabe am: \submissiondate \\
    (\semester)
    \vspace*{1.5cm}

    Betreuung durch: \supervisor
  \end{center}
  
\end{titlepage}

%Seitenränder restaurieren
\restoregeometry
\newpage
	
%Leere Seite einfügen
\mbox{}
\newpage
	
%Seitenangaben aktivieren
\pagestyle{fancy}
	
%Vorverzeichnisse einbinden
\begingroup

% Header auch für Seiten ohne Kopfzeile einfügen
\fancypagestyle{plain}

% Zeilenabstand ändern
\singlespacing

% Inhaltsverzeichnis einfügen
\tableofcontents
\newpage

% Abbildungsverzeichnis einfügen
\listoffigures
\newpage

% Tabellenverzeichnis einfügen
\listoftables
\newpage

% Quellcodeverzeichnis
\renewcommand{\lstlistoflistings}{\begingroup
\tocfile{\lstlistingname}{lol}
\endgroup}
\renewcommand{\lstlistingname}{Quellcodeverzeichnis}
\lstlistoflistings
\newpage

% Abkürzungsverzeichnis einfügen
\input{preindex/abbreviations}
\newpage

% Symbolverzeichnis erstellen
\section*{Symbolverzeichnis}
\markboth{Symbolverzeichnis}{Symbolverzeichnis}
\phantomsection
\addcontentsline{toc}{section}{\protect Symbolverzeichnis}

\begin{acronym}
	\itemsep-20pt
	% Hier Abkürzungen eintragen mit \acro{Kürzel}[Kurzform]{Langform} bzw. \acroplural{Kürzel}[Kurzform]{Langform}
	% Wird die Pluralform benötigt, so muss unter dem selben Kürzel auch die Singularform hinterlegt sein!
	% Im Text kann dann \ac{Kürzel} bzw. \acp{Kürzel} verwendet werden.
	% =============================================================================================================
	\acro{a}[\ensuremath{A}]{{\acrounit{\meter^2}Oberfläche}}
\end{acronym}
\newpage

% Formelverzeichnis erstellen
\newlistof{equations}{equ}{Formelverzeichnis}
\renewcommand{\listofequations}{\begingroup
\tocfile{Formelverzeichnis}{equ}
\endgroup}
\listofequations
\newpage

\endgroup

%Testtext
\section{Beispiele}\label{sec:beispiel}
	
Lorenawdawdwadz awod awd aiwd awdu awoud awduioawud oiawudoawuoduawoduawoidiuawduawodu awd awd awudwad uawdawiduawdawd adwaddwda\footnote{Testfußnote}

\subsection{Formeln}\label{subsec:formulas}

\begin{equation}\label{equ:relativitätstheorie}
	x^2 + mx + t = 0
\end{equation}
\equationentry{Geradengleichung}

Siehe \autoref{equ:relativitätstheorie}
	
\subsection{Bilder/Abbildungen}\label{subsec:abbildungen}
	
Einzelnes Bild
	
\begin{figure}[H]
	\centering
	\begin{minipage}{0.6\textwidth}
		\centering
		\fbox{\includegraphics[width=\textwidth]{logo}}
		\sourceright{Hier cite einfügen}
	\end{minipage}
	\caption{Logo der Hochschule München}
  	\label{fig:logo}
\end{figure}
	
Subfigures siehe \autorefpage{fig:coffee}
	
\begin{figure}[H]
  	\centering
  	\begin{subfigure}[b]{0.4\textwidth}
    	\includegraphics[width=\textwidth]{logo}
    	\caption{A logo.}
  	\end{subfigure}
  	\begin{subfigure}[b]{0.4\textwidth}
    	\includegraphics[width=\textwidth]{logo}
    	\caption{More logo.}
  	\end{subfigure}
  	\caption{The same cup of coffee. Two times.}
  	\label{fig:coffee}
\end{figure}
	
Unabhängige Bilder (aber nebeneinander)
	
\begin{figure}[H]
	\centering
	\begin{minipage}{.5\textwidth}
  		\centering
  		\includegraphics[width=\textwidth]{logo}
  		\caption{A figure}
  		\label{fig:test1}
	\end{minipage}%
	\begin{minipage}{.5\textwidth}
  		\centering
  		\includegraphics[width=\textwidth]{logo}
  		\caption{figure}
  		\label{fig:test2}
	\end{minipage}
\end{figure}
	
\subsection{Aufzählungen}\label{subsec:lists}
	
Aufzählungen ohne voranstehende Beschriftung
	
\begin{description}
	\item{Item 1}
	\item{Item 2}
	\item{Item 3}
\end{description}
	
Ungeordnete Aufzählungen
	
\begin{itemize}
	\item{Item 1}
	\item{Item 2}
	\item{Item 3}
\end{itemize}
	
Geordnete Aufzählungen 
	
\begin{enumerate} %[a/1/I/A] optional
	\item{Item 1}
	\begin{enumerate}
		\item Auch verschachtelt möglich
		\item Auch verschachtelt möglich
	\end{enumerate}
	\item{Item 2}
	\item{Item 3}
\end{enumerate}
	
Weitere Umgebungen sind compactenum und compactitem
	
\subsection{Tabellen}\label{subsec:tables}
	
Im folgenden eine Tabelle mit 2 Spalten und mehreren Zeilen sowie einer alternierenden Zeilenfarbe
	
\begin{table}[H]
	\centering
	\caption{Beispieltabelle}
	\label{tab:test}
	\rowcolors{2}{gray!25}{white}
	\begin{tabular}{|c|c|}
    	\rowcolor{gray!50}
    	\hline
    	Table head & Table head\\
    	Some values & Some values\\
    	Some values & Some values\\
    	Some values & Some values\\
    	Some values & Some values\\
    	\hline
  	\end{tabular}
	\sourcecenter{Das ist einfach nur ein Test}
\end{table}
	
\subsection{Abkürzungen}\label{subsec:abbrev}
	
Das ist ein \ac{html}
	
\newpage
	
Das ist immer noch ein \ac{html}. \acp{dr} sind auch ganz toll.
	
\subsection{Source-Code}\label{subsec:code}
	
\begin{lstlisting}[style=htmlcssjs,float=h,caption={JavaScript Hello World Beispiel},label=lst:hello]
var x = "Hello World" //Das ist ein einfacher String
console.log(x.toUppercase());
/* Dieser Code ist sehr cool aber
effektiv bringt er eigentlich nichts. Hähäh */
\end{lstlisting}
	
\autoref{lst:hello} ist ziemlich cool.

\begin{lstlisting}[style=htmlcssjs,float=h,caption={Einfaches HTML Beispiel},label=lst:html]
<!DOCTYPE html>
<html>
	<body class="Hallo">
		<h1>My First Heading</h1>
		<p>My first paragraph.</p>
	</body>
</html>
\end{lstlisting}

\includewebcode{Das ist ein Beispiel}{test.html}

Ich versuche jetzt mal ein autoref auf \autoref{lst:test.html}
	
\end{document}