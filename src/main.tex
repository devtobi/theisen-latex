% Import der Config-Variablen
%% -- Allgemeine Konfiguration -- %%

% Textgröße
\newcommand{\maintextsize}{12pt}

% Papierart
\newcommand{\papersize}{a4paper}

% Ränder
\newcommand{\leftpapermargin}{4cm}
\newcommand{\rightpapermargin}{2cm}
\newcommand{\toppapermargin}{3.8cm}
\newcommand{\bottompapermargin}{2cm}

%% -- Deckblattkonfiguration -- %%

% Rand
\newcommand{\titlemargin}{2cm}

% Name der Universität/Hochschule
\newcommand{\universityname}{Universitätsname}

% Name der Fakultät
\newcommand{\facultyname}{Fakultätsname}

% Dateiname des Logos
\newcommand{\logofilename}{logo}

% Arbeitstyp
\newcommand{\thesistype}{Bachelorarbeit}	

% Titel der Arbeit
\newcommand{\thesistitle}{Titel}

% Zielabschluss
\newcommand{\studygoal}{Bachelor of Science}

% Autor der Arbeit
\newcommand{\thesisauthor}{Autor}

% Studiengang
\newcommand{\studytype}{Studiengang}

% Matrikelnummer
\newcommand{\studentnumber}{Matrikelnummer}

% Semesternummer
\newcommand{\semesternumber}{Fachsemester}

% Mail
\newcommand{\mail}{.edu-Mail}

% Abgabetermin
\newcommand{\submissiondate}{Abgabetermin}

% Semester
\newcommand{\semester}{WS 2020}

% Betreuer (mit akademischen Titeln)
\newcommand{\supervisor}{Betreuer}


%% -- Inhaltsverzeichnis - Konfiguration -- %%

% Einzug Subsection
\newcommand{\idxsubsecindent}{0.25cm}

% Einzug Subsubsection
\newcommand{\idxsubsubsecindent}{0.6cm}


%% -- Vorwort - Konfiguration -- %%
% Ändern Sie den Namen des Abschnitts auf 'Vorbemerkung' wenn Sie sich sehr kurz halten möchten.
\newcommand{\preamblename}{Vorwort}

% Verfassungsort
\newcommand{\preamblelocation}{München}

% Verfassungsdatum
\newcommand{\preambledate}{im Winter 2020}


% Import der Befehle
% Befehl um Textbox für den Druck zu generieren
\newcommand{\calibrationbox}[2]{% parameters: #1=width, #2=height
	\setlength{\unitlength}{1.0mm}%
	\begin{picture}(#1,#2)%
	\linethickness{0.05mm}%
	\put(0,0){\dashbox{0.2}(#1,#2)%
	{\parbox{#1mm}{%
	\centering\footnotesize 
	width $ = #1 \textrm{mm}$\\
	height $ = #2 \textrm{mm}$
	}}}\end{picture}
}

% Eigener Befehl für autoref + Seitenangabe
\newcommand{\autorefpage}[1]{\autoref{#1} (S. \pageref{#1})}

% Befehl für Quellenangaben direkt unter dem Bild (rechts)
\newcommand*{\sourceright}[1]{\par\raggedleft\footnotesize \textit{Quelle}:~#1}

% Befehl für Quellenangaben direkt unter dem Bild (zentriert)
\newcommand*{\sourcecenter}[1]{\par\centering\footnotesize \textit{Quelle}:~#1}

% Festlegen der Dokumentenklasse
\documentclass[\maintextsize, \papersize, fleqn, xcolor=dvipsnames,chapterprefix]{scrartcl}

% Metadatenangaben
\author{\thesisauthor}
\title{\thesistitle}

% Einbinden der Abhängigkeiten
%%% === BEGINN DER KONFIGURATION === %%%

% Stichwortverzeichnis
\usepackage{makeidx}
\makeindex
\renewcommand{\indexname}{Stichwortverzeichnis}

% Silbentrennung
\hyphenation{Au-dio-de-skrip-ti-on Sprach-er-ken-nung Ver-sions-sprung}

% PDF-Inhalt such- und kopierbar machen
\usepackage[resetfonts]{cmap}

% Querformat erlauben
\usepackage{pdflscape}

% Textschriftart ändern
\usepackage{lmodern}

% Anpassungen für deutsche Sprache + englische Sprache
\usepackage[utf8]{inputenc}
\usepackage[T1]{fontenc}
\usepackage[ngerman,english]{babel}

% Mehr Farben zulassen
\usepackage[table,dvipsnames]{xcolor}
%\usepackage{color}
\definecolor{lightgray}{rgb}{0.95, 0.95, 0.95}
\definecolor{darkgray}{rgb}{0.4, 0.4, 0.4}
\definecolor{editorGray}{rgb}{0.95, 0.95, 0.95}
\definecolor{editorOcher}{rgb}{1, 0.5, 0} % #FF7F00 -> rgb(239, 169, 0)
\definecolor{editorGreen}{rgb}{0, 0.5, 0} % #007C00 -> rgb(0, 124, 0)
\definecolor{orange}{rgb}{1,0.45,0.13}		
\definecolor{olive}{rgb}{0.17,0.59,0.20}
\definecolor{brown}{rgb}{0.69,0.31,0.31}
\definecolor{purple}{rgb}{0.38,0.18,0.81}
\definecolor{lightblue}{rgb}{0.1,0.57,0.7}
\definecolor{lightred}{rgb}{1,0.4,0.5}

% Grafiken
\usepackage{graphicx}
\graphicspath{{images/}}
\usepackage{float}

% Diagramme, Plots uvm.
\usepackage{pgfplots}
\pgfplotsset{compat=1.17}

\usepackage{tikz}
\usepackage{tikz-3dplot}
\usepackage{tikz-cd}
\usepackage{tikz-dependency}
\usepackage{tikz-inet}
\usepackage{tikz-qtree}
\usepackage{tikz-qtree-compat}
\usepackage{tikz-timing}
\usepackage{tikz-timing-overlays}
\usepackage{tikz-timing-nicetabs}
\usepackage{tikz-timing-interval}
\usepackage{tikz-timing-ifsym}
\usepackage{tikz-timing-either}
\usepackage{tikz-timing-counters}
\usepackage{tikz-timing-columntype}
\usepackage{tikz-timing-clockarrows}
\usepackage{tikz-timing-arrows}
\usepackage{tikz-timing-advnodes}
\usetikzlibrary{graphs}
\usetikzlibrary{automata}
\usetikzlibrary{arrows.meta}
\usetikzlibrary{backgrounds}
\usetikzlibrary{calc}
\usetikzlibrary{calendar}
\usetikzlibrary{chains}
\usetikzlibrary{er}
\usetikzlibrary{intersections}
\usetikzlibrary{mindmap}
\usetikzlibrary{matrix}
\usetikzlibrary{patterns}
\usetikzlibrary{plotmarks}
\usetikzlibrary{shapes}
\usetikzlibrary{decorations}
\usetikzlibrary{topaths}
\usetikzlibrary{trees}

%\usetikzlibrary{graphdrawing} Funktioniert nicht mit pdfLaTeX

% Verbessertes Fontrendering
\usepackage[final]{microtype}

% Seitenränder festlegen
\usepackage[\papersize,left=\leftpapermargin,right=\rightpapermargin,top=\toppapermargin,bottom=\bottompapermargin]{geometry}

% 1,5 Zeilenabstand konfigurieren
\usepackage[onehalfspacing]{setspace}

% Zeileneinschub mit Abstand ersetzen
\usepackage{parskip}

% Fußnoten konfigurieren
\usepackage[perpage, flushmargin, hang]{footmisc}
\setlength{\footnotemargin}{4mm}

% Namensnennungen kursiv schreiben
\usepackage{xpatch}
\xpatchcmd{\citeauthor}{\begingroup}{\begingroup\em}{}{}

% Inhaltsverzeichnis Aussehen konfigurieren und ins Inhaltsverzeichnis mit aufnehmen
%\usepackage{tocbibind}
\usepackage[nottoc]{tocbibind}
\usepackage{tocloft}
\renewcommand{\cftsecleader}{\cftdotfill{\cftdotsep}}
\setlength{\cftsubsecindent}{\idxsubsecindent}
\setlength{\cftsubsubsecindent}{\idxsubsubsecindent}
\setlength{\cftbeforesecskip}{3pt}
\setlength{\cftfigindent}{0pt}
\setlength{\cfttabindent}{0pt}

% Verhindern, dass Elemente außerhalb des aktuellen Kapitels platziert werden
\usepackage[section]{placeins}

% Schönere Seitenangaben
\usepackage{fancyhdr}
\fancyhf{}
\fancyhead[EL,OR]{\thepage}
\fancyhead[ER]{\chaptername~\thechapter}
\fancyhead[LO]{\nouppercase{\leftmark}}

% Mathematik-Pakete
\usepackage{amsmath,amsthm,amssymb}
\usepackage{siunitx}
\setlength{\mathindent}{1cm}

% Automatisches Setzen von Anführungszeichen (\enquote, \foreignquote, \blockquote)
\usepackage[autostyle=true,german=quotes,babel]{csquotes}

% Einbinden von PDFs
\usepackage{pdfpages}

% Erlaubt die Erstellung von zusammengesetzten Grafiken
\usepackage{subcaption}

% Leere Anfangs- und Endzeilen vermeiden
\usepackage[defaultlines=4,all]{nowidow}
\widowpenalties 3 10000 10000 150
\clubpenalty 10000

% Bessere Aufzählungen
\usepackage{paralist}

% Bessere Tabellen
\usepackage{booktabs}
\usepackage{tabularx}
\usepackage{ragged2e}
\usepackage{array}
\newcolumntype{L}[1]{>{\RaggedRight\arraybackslash}p{#1}}
\newcolumntype{R}[1]{>{\RaggedLeft\arraybackslash}p{#1}}
\newcolumntype{C}[1]{>{\centering\arraybackslash\hspace{0pt}}p{#1}}

% Euro-Zeichen
\usepackage[official]{eurosym}

% Abkürzungen
\usepackage[footnote, printonlyused]{acronym}

%%% === LISTINGS === %%%
\usepackage{listings}
\usepackage{beramono}
\renewcommand{\lstlistingname}{Quelltext}

% CSS als Sprache definieren
\lstdefinelanguage{CSS}{
  keywords={color,background-image:,margin,padding,font,weight,display,position,top,left,right,bottom,list,style,border,size,white,space,min,width, transition:, transform:, transition-property, transition-duration, transition-timing-function},	
  sensitive=true,
  morecomment=[l]{//},
  morecomment=[s]{/*}{*/},
  morestring=[b]',
  morestring=[b]",
  alsoletter={:},
  alsodigit={-}
}

% JavaScript als Sprache definieren
\lstdefinelanguage{JavaScript}{
  keywords={break, case, catch, continue, debugger, delete, do, else, finally, for, function, if, instanceof, new, return, null, true, false, switch, this, throw, try, typeof, var, void, while, with, let, const},
  ndkeywords={class, export, boolean, throw, implements, import, this},
  sensitive=true,
  morecomment=[l]{//},
  morecomment=[s]{/*}{*/},
  morestring=[b]',
  morestring=[b]"
}

% HTML5 als Sprache definieren
\lstdefinelanguage{HTML5}{
  language=html,
  sensitive=true,	
  alsoletter={<>=-},	
  morecomment=[s]{<!--}{-->},
  tag=[s],
  otherkeywords={
  % General
  >,
  % Standard tags
	<!DOCTYPE,
  </html, <html, <head, <title, </title, <style, </style, <link, </head, <meta, />,
	% body
	</body, <body,
	% Divs
	</div, <div, </div>,
	% headings
	</h1, <h1, </h1>,
	</h2, <h2, </h2>, 
	</h3, <h3, </h3>, 
	</h4, <h4, </h4>, 
	</h5, <h5, </h5>,
	</h6, <h6, </h6>,  
	% Paragraphs
	</p, <p, </p>,
	% scripts
	</script, <script,
  % More tags...
  <canvas, /canvas>, <svg, <rect, <animateTransform, </rect>, </svg>, <video, <source, <iframe, </iframe>, </video>, <image, </image>, <header, </header, <article, </article, <strong, </strong, <a, </a, <nav, </nav, <main, </main, <footer, </footer, <img, </img, <figure, </figure, <figcaption, </figcaption, <track, </video, <label, </label, <input, <audio, </audio, <fieldset, </fieldset, <legend, </legend, <span, </span, <form, </form, <br, <kbd, </kbd, <li, </li, <ul, </ul, <link, </link
  },
  ndkeywords={
  % General
  =,
  % HTML attributes
  charset=, src=, id=, width=, height=, style=, type=, rel=, href=,
  % SVG attributes
  fill=, attributeName=, begin=, dur=, from=, to=, poster=, controls=, x=, y=, repeatCount=, xlink:href=,
  % properties
  margin:, padding:, background-image:, border:, top:, left:, position:, width:, height:, margin-top:, margin-bottom:, font-size:, line-height:,
	% CSS3 properties
  transform:, -moz-transform:, -webkit-transform:,
  animation:, -webkit-animation:,
  transition:,  transition-duration:, transition-property:, transition-timing-function:,
  }
}

\lstdefinestyle{htmlcssjs} {%
  % General design
%  backgroundcolor=\color{editorGray},
  basicstyle={\footnotesize\ttfamily},   
  frame=single,
  captionpos=b,
  % line-numbers
  xleftmargin={0.75cm},
  numbers=left,
  stepnumber=1,
  firstnumber=1,
  numberfirstline=true,	
  % Code design
  identifierstyle=\color{black},
  keywordstyle=\color{blue}\bfseries,
  ndkeywordstyle=\color{editorGreen}\bfseries,
  stringstyle=\color{editorOcher}\ttfamily,
  commentstyle=\color{brown}\ttfamily,
  numberstyle=\color{Gray},
  rulesepcolor=\color{lightgray},
  % Code
  language=HTML5,
  alsolanguage=JavaScript,
  alsodigit={.:;},	
  tabsize=2,
  showtabs=false,
  showspaces=false,
  showstringspaces=false,
  extendedchars=true,
  breaklines=true,
  breakatwhitespace=false,
  columns=flexible,
  keepspaces=false,
  inputencoding=utf8,
  resetmargins=true,
  % German umlauts
  literate=
  {á}{{\'a}}1 {é}{{\'e}}1 {í}{{\'i}}1 {ó}{{\'o}}1 {ú}{{\'u}}1
  {Á}{{\'A}}1 {É}{{\'E}}1 {Í}{{\'I}}1 {Ó}{{\'O}}1 {Ú}{{\'U}}1
  {à}{{\`a}}1 {è}{{\`e}}1 {ì}{{\`i}}1 {ò}{{\`o}}1 {ù}{{\`u}}1
  {À}{{\`A}}1 {È}{{\'E}}1 {Ì}{{\`I}}1 {Ò}{{\`O}}1 {Ù}{{\`U}}1
  {ä}{{\"a}}1 {ë}{{\"e}}1 {ï}{{\"i}}1 {ö}{{\"o}}1 {ü}{{\"u}}1
  {Ä}{{\"A}}1 {Ë}{{\"E}}1 {Ï}{{\"I}}1 {Ö}{{\"O}}1 {Ü}{{\"U}}1
  {â}{{\^a}}1 {ê}{{\^e}}1 {î}{{\^i}}1 {ô}{{\^o}}1 {û}{{\^u}}1
  {Â}{{\^A}}1 {Ê}{{\^E}}1 {Î}{{\^I}}1 {Ô}{{\^O}}1 {Û}{{\^U}}1
  {Ã}{{\~A}}1 {ã}{{\~a}}1 {Õ}{{\~O}}1 {õ}{{\~o}}1
  {œ}{{\oe}}1 {Œ}{{\OE}}1 {æ}{{\ae}}1 {Æ}{{\AE}}1 {ß}{{\ss}}1
  {ű}{{\H{u}}}1 {Ű}{{\H{U}}}1 {ő}{{\H{o}}}1 {Ő}{{\H{O}}}1
  {ç}{{\c c}}1 {Ç}{{\c C}}1 {ø}{{\o}}1 {å}{{\r a}}1 {Å}{{\r A}}1
  {€}{{\euro}}1 {£}{{\pounds}}1 {«}{{\guillemotleft}}1
  {»}{{\guillemotright}}1 {ñ}{{\~n}}1 {Ñ}{{\~N}}1 {¿}{{?`}}1
}

\lstdefinestyle{htmlcssjs_appendix} {%
  % General design
%  backgroundcolor=\color{editorGray},
  basicstyle={\tiny\ttfamily},   
  frame=single,
  captionpos=b,
  % line-numbers
  xleftmargin={0.75cm},
  numbers=left,
  stepnumber=1,
  firstnumber=1,
  numberfirstline=true,	
  % Code design
  identifierstyle=\color{black},
  keywordstyle=\color{blue}\bfseries,
  ndkeywordstyle=\color{editorGreen}\bfseries,
  stringstyle=\color{editorOcher}\ttfamily,
  commentstyle=\color{brown}\ttfamily,
  numberstyle=\color{Gray},
  rulesepcolor=\color{lightgray},
  % Code
  language=HTML5,
  alsolanguage=JavaScript,
  alsodigit={.:;},	
  tabsize=2,
  showtabs=false,
  showspaces=false,
  showstringspaces=false,
  extendedchars=true,
  breaklines=true,
  breakatwhitespace=false,
  columns=flexible,
  keepspaces=false,
  inputencoding=utf8,
  resetmargins=true,
  % German umlauts
  literate=
  {á}{{\'a}}1 {é}{{\'e}}1 {í}{{\'i}}1 {ó}{{\'o}}1 {ú}{{\'u}}1
  {Á}{{\'A}}1 {É}{{\'E}}1 {Í}{{\'I}}1 {Ó}{{\'O}}1 {Ú}{{\'U}}1
  {à}{{\`a}}1 {è}{{\`e}}1 {ì}{{\`i}}1 {ò}{{\`o}}1 {ù}{{\`u}}1
  {À}{{\`A}}1 {È}{{\'E}}1 {Ì}{{\`I}}1 {Ò}{{\`O}}1 {Ù}{{\`U}}1
  {ä}{{\"a}}1 {ë}{{\"e}}1 {ï}{{\"i}}1 {ö}{{\"o}}1 {ü}{{\"u}}1
  {Ä}{{\"A}}1 {Ë}{{\"E}}1 {Ï}{{\"I}}1 {Ö}{{\"O}}1 {Ü}{{\"U}}1
  {â}{{\^a}}1 {ê}{{\^e}}1 {î}{{\^i}}1 {ô}{{\^o}}1 {û}{{\^u}}1
  {Â}{{\^A}}1 {Ê}{{\^E}}1 {Î}{{\^I}}1 {Ô}{{\^O}}1 {Û}{{\^U}}1
  {Ã}{{\~A}}1 {ã}{{\~a}}1 {Õ}{{\~O}}1 {õ}{{\~o}}1
  {œ}{{\oe}}1 {Œ}{{\OE}}1 {æ}{{\ae}}1 {Æ}{{\AE}}1 {ß}{{\ss}}1
  {ű}{{\H{u}}}1 {Ű}{{\H{U}}}1 {ő}{{\H{o}}}1 {Ő}{{\H{O}}}1
  {ç}{{\c c}}1 {Ç}{{\c C}}1 {ø}{{\o}}1 {å}{{\r a}}1 {Å}{{\r A}}1
  {€}{{\euro}}1 {£}{{\pounds}}1 {«}{{\guillemotleft}}1
  {»}{{\guillemotright}}1 {ñ}{{\~n}}1 {Ñ}{{\~N}}1 {¿}{{?`}}1
}

\lstset{
%	breaklines=true,
%	breakatwhitespace=false,
%	captionpos=b,
%	frame=shadowbox,
%	columns=flexible,
%	keepspaces=false,
%	numbers=left,
%	showspaces=false,
%	showstringspaces=false,
%	showtabs=false,
%	identifierstyle=\color{black},
%	ndkeywordstyle=\color{MidnightBlue}\bfseries,
%	basicstyle=\footnotesize\ttfamily,
%	numberstyle=\color{Gray},
%	keywordstyle=\bfseries\color{RoyalBlue},
%	commentstyle=\color{ForestGreen},
%	stringstyle=\color{Orange},
%	xleftmargin={0.75cm},
%	tabsize=2,
%	extendedchars=true,
%	inputencoding=utf8,
%	basicstyle=\ttfamily
%	resetmargins=true,
%	rulesepcolor=\color{lightgray},
}

\lstset{basicstyle=\ttfamily}

%%% === ENDE LISTINGS === %%%

% Beschriftung von Bildern/Tabellen/Listings usw.
\usepackage[
	format=plain,
	margin=0.5cm,
	labelformat=simple,
	labelfont={bf},
	font=footnotesize,
	justification=centering
]{caption}
\usepackage{chngcntr}
\counterwithin{figure}{section}
\counterwithin{table}{section}
\counterwithin{equation}{section}
\AtBeginDocument{\counterwithin{lstlisting}{section}}

% Literaturverzeichnis & Harvard-Stil einrichten
% Chicago-Stil
%\usepackage[
%authordate,
%backend=biber,
%bibencoding=inputenc,
%language=ngerman,
%footmarkoff,
%alwaysrange=true,
%natbib,
%]{biblatex-chicago}
%\renewcommand{\postnotedelim}{%
%  \iffieldpages{postnote}
%    {\addcomma\space S.}
%    {\addspace}}

% Harvard-Stil
\usepackage[style=authoryear,backend=biber]{biblatex}
\renewcommand*{\finentrypunct}{}
\setlength\bibitemsep{0pt}
\setlength\bibinitsep{10pt}
\renewcommand{\mkbibnamelast}[1]{\mkbibemph{#1}}
\renewcommand{\mkbibnamefirst}[1]{\mkbibemph{#1}}
\renewcommand*{\nameyeardelim}{\addcomma\space}
\renewcommand{\UrlFont}{\footnotesize\ttfamily}

\addbibresource{bibliography.bib}

%Verlinkungen und PDF Daten initialisieren
\usepackage{hyperref}
\hypersetup{
	pdftitle = {\thesistitle},
	pdfauthor = {\thesisauthor},
	pdfcreator = {pdflatex},
	pdfkeywords={\thesistype, \thesistitle},
	pdfsubject={\thesistype},
	pdfproducer = {LaTeX with hyperref},
	colorlinks = false,
	linktoc = all,
	linkbordercolor = {0 0 0.5},
}
\usepackage[all]{hypcap}

% Autoref Bezeichnungen ändern
\addto\extrasngerman{%
  \def\sectionautorefname{Kapitel}%
}
\addto\extrasngerman{%
  \def\subsectionautorefname{Abschnitt}%
}
\addto\extrasngerman{%
  \def\subsubsectionautorefname{Unterabschnitt}%
}

%%% === ENDE DER KONFIGURATION === %%%

%%% === BEGINN DES DOKUMENTS === %%%
\begin{document}

%Auf Deutsch stellen
\selectlanguage{ngerman}

%Namen der autoref generierten Verweise ändern
\renewcommand{\sectionautorefname}{Kapitel}
\renewcommand{\subsectionautorefname}{Abschnitt}
\renewcommand{\subsubsectionautorefname}{Unterabschnitt}

%Seitennummerierung abschalten
\pagestyle{empty}
	
%Deckblatt einbinden
%Seitenränder abschalten
\newgeometry{left=\titlemargin,right=\titlemargin,top=\titlemargin,bottom=\titlemargin}

\begin{titlepage}

   \singlespacing

   %Inhalt zentrieren
   \begin{center} \large 
    
    \universityname
    \vspace*{0.5cm}
    
    \facultyname
    \vspace*{1.5cm}
    
	\includegraphics[scale=0.45]{titlepage/\logofilename}
    \vspace*{2cm} 

	\thesistype \ zum Thema
	\vspace*{0.5cm}
	
    {\huge\textbf{\thesistitle}}
    \vspace*{1.5cm}
    
    
  	Zur Erlangung des Grades \studygoal
  	\vspace*{1.5cm}

    \textbf{\thesisauthor}
    \vspace*{0.5cm}
    
    Studiengang: \studytype \\
    Matrikelnummer: \studentnumber \\
    Fachsemester: \semesternumber \\
    E-Mail: \mail \\
    \vspace*{2.0cm}

    Abgabe am: \submissiondate \\
    (\semester)
    \vspace*{1.5cm}

    Betreuung durch: \supervisor
  \end{center}
  
\end{titlepage}

%Seitenränder restaurieren
\restoregeometry
\newpage
	
%Leere Seite einfügen (Hier könnte z.B die handschriftliche Widmung stehen)
\mbox{}
\newpage

%Sperrvermerk einbinden
%% DEUTSCH

\section*{\centering{Sperrvermerk}}

Die vorliegende \thesistype \ mit dem Titel:
\begin{center}
	\textit{\thesistitle}
\end{center}
beinhaltet interne und vertrauliche Informationen des Unternehmens:
\begin{center}
	\textit{\companyname}
\end{center}

\textbf{\\
Eine Einsicht in diese \thesistype \ ist nicht gestattet. Ausgenommen davon sind die betreuenden Dozenten sowie die befugten Mitglieder des Prüfungsausschusses. Eine Veröffentlichung und Vervielfältigung der \thesistype \ - auch in Auszügen - ist nicht gestattet.
}

Ausnahmen von dieser Regelung bedürfen einer Genehmigung des Unternehmens \textit{\companyname}.

\rule{\textwidth}{0.4pt}

%% ENGLISH

\selectlanguage{english}
\section*{\centering{Confidentiality Clause}}

This \thesistypeenglish \ with title:
\begin{center}
	\textit{\thesistitle}
\end{center}
contains confidential data of company:
\begin{center}
	\textit{\companyname}
\end{center}

\textbf{\\
This work may only be made available to the first and second reviewers and authorized members of the board of examiners. Any publication and duplication of this \thesistypeenglish \ - even in part - is prohibited.
}

An inspection of this work by third parties requires the expressed permission of the company \textit{\companyname}.

\selectlanguage{ngerman}
	
%Seitenangaben aktivieren
\pagestyle{fancy}
	
%Vorverzeichnisse einbinden
\begingroup

% Header auch für Seiten ohne Kopfzeile einfügen
\fancypagestyle{plain}

% Zeilenabstand ändern
\singlespacing

% Inhaltsverzeichnis einfügen
\tableofcontents
\newpage

% Abbildungsverzeichnis einfügen
\listoffigures
\newpage

% Tabellenverzeichnis einfügen
\listoftables
\newpage

% Quellcodeverzeichnis
\renewcommand{\lstlistoflistings}{\begingroup
\tocfile{\lstlistingname}{lol}
\endgroup}
\renewcommand{\lstlistingname}{Quellcodeverzeichnis}
\lstlistoflistings
\newpage

% Abkürzungsverzeichnis einfügen
\input{preindex/abbreviations}
\newpage

% Symbolverzeichnis erstellen
\section*{Symbolverzeichnis}
\markboth{Symbolverzeichnis}{Symbolverzeichnis}
\phantomsection
\addcontentsline{toc}{section}{\protect Symbolverzeichnis}

\begin{acronym}
	\itemsep-20pt
	% Hier Abkürzungen eintragen mit \acro{Kürzel}[Kurzform]{Langform} bzw. \acroplural{Kürzel}[Kurzform]{Langform}
	% Wird die Pluralform benötigt, so muss unter dem selben Kürzel auch die Singularform hinterlegt sein!
	% Im Text kann dann \ac{Kürzel} bzw. \acp{Kürzel} verwendet werden.
	% =============================================================================================================
	\acro{a}[\ensuremath{A}]{{\acrounit{\meter^2}Oberfläche}}
\end{acronym}
\newpage

% Formelverzeichnis erstellen
\newlistof{equations}{equ}{Formelverzeichnis}
\renewcommand{\listofequations}{\begingroup
\tocfile{Formelverzeichnis}{equ}
\endgroup}
\listofequations
\newpage

\endgroup

%Vortexte einbinden
%Einfügen eines Mottos, Sprichtworts, Zitat einer Persönlichkeit
\pagestyle{empty}
\pagestyle{empty}
\newgeometry{left=\leftpapermargin,right=\rightpapermargin,top=\titlemargin,bottom=\titlemargin}
\vspace*{\fill}
\begin{center}
	\huge{
	\glqq
	\textit{Das ist ein ganz wunderbares Zitat und beschreibt diese Arbeit sehr gut.}\grqq\\
	\textbf{--- Albert Einstein ---}
	}
\end{center}

\vfill
\newpage

% Einfügen eines Vorworts
\pagestyle{fancy}
\markboth{\preamblename}{\preamblename}
\phantomsection
\addcontentsline{toc}{section}{\protect \preamblename}
% Dieses Dokument dient als Vorwort der Arbeit (wird aber meistens nur bei zu veröffentlichenden Arbeiten benötigt)
\section*{\preamblename}\label{sec:\preamblename}

% Beginn

Hier kann zum Beispiel auf den Anlass und die Bedeutung der Schrift und deren Stellung in der wissenschaftlichen Diskussion eingegangen werden.
Außerdem kann auf die sachliche und/oder finanzielle Unterstützung durch Dritte eingegangen werden.
Des Weiteren sollte auf Umfang und Quelle der erteilten Druckerlaubnisse hingewiesen und Dank ausgesprochen werden.

Art und Umfang der persönlichen Mitarbeit sowie Unterstützung durch Dritte ist zu erwähnen.

% Ende

\vspace{0.5cm}
\begingroup
\raggedright{\preamblelocation, \preambledate}
\hfill
\raggedleft{\textit{\thesisauthor}}
\endgroup

\newpage
\newpage

%Text einbinden
% 1. Einleitung
\section{Einleitung}\label{sec:einleitung}

Das ist das einleitende Kapitel dieser Arbeit.
\newpage

% 2. Beispiele
\section{Beispiele}\label{sec:beispiele}

Lorenawdawdwadz awod awd aiwd awdu awoud awduioawud oiawudoawuoduawoduawoidiuawduawodu awd awd awudwad uawdawiduawdawd adwaddwda\footnote{Testfußnote}

\subsection{Formeln}\label{subsec:formulas}

\begin{equation}\label{equ:gerade}
	x^2 + mx + t = 0
\end{equation}
\equationentry{Geradengleichung}

Siehe \autoref{equ:gerade}

\subsection{Bilder/Abbildungen}\label{subsec:abbildungen}
	
Einzelnes Bild.\autocite{higgins1996activation}
	
\begin{figure}[H]
	\centering
	\begin{minipage}{0.6\textwidth}
		\centering
		\fbox{\includegraphics[width=\textwidth]{2_beispiele/hm}}
		\sourceright{Hier die Quelle einfügen}
	\end{minipage}
	\caption{Logo der Hochschule München}
  	\label{fig:logo}
\end{figure}
	
Subfigures siehe \autorefpage{fig:coffee}
	
\begin{figure}[H]
  	\centering
  	\begin{subfigure}[b]{0.4\textwidth}
    	\includegraphics[width=\textwidth]{example-image}
    	\sourceright{Hier die Quelle einfügen}
    	\caption{A logo.}
  	\end{subfigure}
  	\begin{subfigure}[b]{0.4\textwidth}
    	\includegraphics[width=\textwidth]{example-image}
    	\sourceplaceholder
    	\caption{More logo.}
  	\end{subfigure}
  	\caption{The same cup of coffee. Two times.}
  	\label{fig:coffee}
\end{figure}
	
Unabhängige Bilder (aber nebeneinander)
	
\begin{figure}[H]
	\centering
	\begin{minipage}{.5\textwidth}
  		\centering
  		\includegraphics[width=\textwidth]{example-image}
  		\caption{A figure}
  		\label{fig:test1}
	\end{minipage}%
	\begin{minipage}{.5\textwidth}
  		\centering
  		\includegraphics[width=\textwidth]{example-image}
  		\caption{figure}
  		\label{fig:test2}
	\end{minipage}
\end{figure}

\subsection{Aufzählungen}\label{subsec:lists}
	
Aufzählungen ohne voranstehende Beschriftung
	
\begin{description}
	\item{Item 1}
	\item{Item 2}
	\item{Item 3}
\end{description}
	
Ungeordnete Aufzählungen
	
\begin{itemize}
	\item{Item 1}
	\item{Item 2}
	\item{Item 3}
\end{itemize}
	
Geordnete Aufzählungen 
	
\begin{enumerate} %[a/1/I/A] optional
	\item{Item 1}
	\begin{enumerate}
		\item Auch verschachtelt möglich
		\item Auch verschachtelt möglich
	\end{enumerate}
	\item{Item 2}
	\item{Item 3}
\end{enumerate}
	
Weitere Umgebungen sind compactenum und compactitem

\subsection{Tabellen}\label{subsec:tables}
	
Im folgenden eine Tabelle mit 2 Spalten und mehreren Zeilen sowie einer alternierenden Zeilenfarbe.
	
\begin{table}[H]
	\centering
	\caption{Beispieltabelle}
	\label{tab:test}
	\rowcolors{2}{gray!25}{white}
	\begin{tabular}{|c|c|}
    	\rowcolor{gray!50}
    	\hline
    	Table head & Table head\\
    	Some values & Some values\\
    	Some values & Some values\\
    	Some values & Some values\\
    	Some values & Some values\\
    	\hline
  	\end{tabular}
	\sourcecenter{Hier könnte die Quelle der Tabelle stehen}
\end{table}

\subsection{Abkürzungen}\label{subsec:abbrev}
	
Das ist ein \ac{html}
	
\newpage
	
Das ist immer noch ein \ac{html}. \acp{dr} sind auch ganz toll.

\subsection{Source-Code}\label{subsec:code}
	
\begin{lstlisting}[style=htmlcssjs,float=h,caption={JavaScript Hello World Beispiel},label=lst:hello]
var x = "Hello World" //Das ist ein einfacher String
console.log(x.toUppercase());
/* Dieser Code ist sehr cool aber
effektiv bringt er eigentlich nichts. Hähäh */
\end{lstlisting}
	
\autoref{lst:hello} ist ziemlich cool.

\begin{lstlisting}[style=htmlcssjs,float=h,caption={Einfaches HTML Beispiel},label=lst:html]
<!DOCTYPE html>
<html>
	<body class="Hallo">
		<h1>My First Heading</h1>
		<p>My first paragraph.</p>
	</body>
</html>
\end{lstlisting}

\includewebcode{Das ist ein Beispiel}{test.html}

Ich versuche jetzt mal ein autoref auf \autoref{lst:test.html}
\newpage

%Anhang einbinden
\section*{Anhang}
\appendix

% Zähler zurücksetzen und Captionbeschriftung ändern
\setcounter{figure}{0}
\setcounter{table}{0}
\setcounter{equation}{0}
\setcounter{lstlisting}{0}
\renewcommand{\thefigure}{\Roman{figure}}
\renewcommand{\thetable}{\Roman{table}}
\renewcommand{\theequation}{\Roman{equation}}
\renewcommand{\thelstlisting}{\Roman{lstlisting}}

\section{Allgemeine Ergänzungen}

\begin{figure}[H]
	\centering
	\begin{minipage}{0.6\textwidth}
		\centering
		\fbox{\includegraphics[width=\textwidth]{2_beispiele/hm}}
		\sourceright{Hier die Quelle einfügen}
	\end{minipage}
	\caption{Logo der Hochschule München}
  	\label{fig:logo_anhang}
\end{figure}
		
%Nachtexte einbinden
% Einfügen eines Nachworts
%\markboth{Nachwort}{Nachwort}
%\phantomsection
%\addcontentsline{toc}{section}{\protect Nachwort}
%% Dieses Dokument dient als Nachwort der Arbeit (wird aber meistens nur bei zu veröffentlichenden Arbeiten benötigt)
\section*{Nachwort}\label{sec:nachwort}

% Beginn

Hier können Gedanken des Autors stehen, die nur einen Teil der Leser ansprechen oder aber persönliche Zusammenhänge aufzeigen.
Liegt zwischen Erstellung und Veröffentlichung ein erheblicher Zeitraum, so kann auf themenspezifische Entwicklungen oder weitere Forschungsergebnisse eingegeangen werden.
Außerdem kann auf die sachliche und/oder finanzielle Unterstützung durch Dritte eingegangen werden.
Des Weiteren sollte auf Umfang und Quelle der erteilten Druckerlaubnisse hingewiesen und Dank ausgesprochen werden.

% Ende

\vspace{0.5cm}

\begingroup
\raggedright{\epiloguelocation, \epiloguedate}
\hfill
\raggedleft{\textit{\thesisauthor}}
\raggedright
\endgroup
%\newpage

%Nachverzeichnisse einbinden
%Literaturverzeichnis erstellen
\printbibliography[notkeyword=Quelle,heading=bibintoc]
\newpage

%Quellenverzeichnis erstellen
\printbibliography[keyword=Quelle,title={Quellenverzeichnis},heading=bibintoc]
\newpage

%Stichwortverzeichnis erstellen
%\addcontentsline{toc}{section}{Stichwortverzeichnis}
%\printindex
%\newpage

%Eidesstattliche Erklärung einbinden	
\pagestyle{empty}
\section*{Eidesstattliche Erklärung}

Ich erkläre hiermit an Eides Statt, dass ich die vorliegende Arbeit selbstständig und ohne Benutzung anderer als der angegebenen Hilfsmittel angefertigt habe; die aus fremden Quellen (einschließlich elektronischer Quellen) direkt oder indirekt übernommenen Gedanken sind ausnahmslos als solche kenntlich gemacht.

% Nur notwendig falls Unterstützungsleistungen in Anspruch genommen wurden

%Bei der Auswahl und Auswertung des Materials sowie bei der Erstellung des Textes habe ich Unterstützungsleistungen von folgenden Personen erhalten:
%\begin{compactenum}
%	\item Person 1
%	\item Person 2
%	\item ...
%\end{compactenum}

Weitere Personen waren an der geistigen Leistung der Arbeit nicht beteiligt. Ich habe nicht die Hilfe eines Beraters oder wissenschaftlichen Coaches in Anspruch genommen. Dritte haben von mir weder unmittelbar noch mittelbar Geld oder geldwerte Leistungen für Arbeiten erhalten, die im Zusammenhang mit dem Inhalt der Arbeit stehen.

Die Arbeit wurde bisher weder im Inland noch im Ausland ganz oder in Teilen einer anderen Prüfungsbehörde vorgelegt und auch noch nicht gedruckt oder elektronisch veröffentlicht.

Ich bin mir bewusst, dass eine falsche Erklärung rechtliche Folgen haben wird.

\vspace{1.0cm}

\begingroup
\raggedright{\underline{\hspace{5.0cm}}}
\hfill
\raggedleft{\underline{\hspace{5.0cm}}}
\endgroup

\vspace{-0.3cm}

\begingroup
\raggedright{Ort, Datum}
\hfill
\raggedleft{Unterschrift}
\endgroup

%Kalibrierungsbox einbinden (Vor dem Binden das entfernen der Seite nicht vergessen)
\calibrationbox{140}{100}
	
%%% === ENDE DES DOKUMENTS === %%%
\end{document}