% Import der Config-Variablen
%% -- Allgemeine Konfiguration -- %%

% Textgröße
\newcommand{\maintextsize}{12pt}

% Papierart
\newcommand{\papersize}{a4paper}

% Ränder
\newcommand{\leftpapermargin}{4cm}
\newcommand{\rightpapermargin}{2cm}
\newcommand{\toppapermargin}{3.8cm}
\newcommand{\bottompapermargin}{2cm}

%% -- Deckblattkonfiguration -- %%

% Rand
\newcommand{\titlemargin}{2cm}

% Name der Universität/Hochschule
\newcommand{\universityname}{Universitätsname}

% Name der Fakultät
\newcommand{\facultyname}{Fakultätsname}

% Dateiname des Logos
\newcommand{\logofilename}{logo}

% Arbeitstyp
\newcommand{\thesistype}{Bachelorarbeit}	

% Titel der Arbeit
\newcommand{\thesistitle}{Titel}

% Zielabschluss
\newcommand{\studygoal}{Bachelor of Science}

% Autor der Arbeit
\newcommand{\thesisauthor}{Autor}

% Studiengang
\newcommand{\studytype}{Studiengang}

% Matrikelnummer
\newcommand{\studentnumber}{Matrikelnummer}

% Semesternummer
\newcommand{\semesternumber}{Fachsemester}

% Mail
\newcommand{\mail}{.edu-Mail}

% Abgabetermin
\newcommand{\submissiondate}{Abgabetermin}

% Semester
\newcommand{\semester}{WS 2020}

% Betreuer (mit akademischen Titeln)
\newcommand{\supervisor}{Betreuer}


%% -- Inhaltsverzeichnis - Konfiguration -- %%

% Einzug Subsection
\newcommand{\idxsubsecindent}{0.25cm}

% Einzug Subsubsection
\newcommand{\idxsubsubsecindent}{0.6cm}


%% -- Vorwort - Konfiguration -- %%
% Ändern Sie den Namen des Abschnitts auf 'Vorbemerkung' wenn Sie sich sehr kurz halten möchten.
\newcommand{\preamblename}{Vorwort}

% Verfassungsort
\newcommand{\preamblelocation}{München}

% Verfassungsdatum
\newcommand{\preambledate}{im Winter 2020}


% Import der Befehle
% Befehl um Textbox für den Druck zu generieren
\newcommand{\calibrationbox}[2]{% parameters: #1=width, #2=height
	\setlength{\unitlength}{1.0mm}%
	\begin{picture}(#1,#2)%
	\linethickness{0.05mm}%
	\put(0,0){\dashbox{0.2}(#1,#2)%
	{\parbox{#1mm}{%
	\centering\footnotesize 
	width $ = #1 \textrm{mm}$\\
	height $ = #2 \textrm{mm}$
	}}}\end{picture}
}

% Eigener Befehl für autoref + Seitenangabe
\newcommand{\autorefpage}[1]{\autoref{#1} (S. \pageref{#1})}

% Befehl für Quellenangaben direkt unter dem Bild (rechts)
\newcommand*{\sourceright}[1]{\par\raggedleft\footnotesize \textit{Quelle}:~#1}

% Befehl für Quellenangaben direkt unter dem Bild (zentriert)
\newcommand*{\sourcecenter}[1]{\par\centering\footnotesize \textit{Quelle}:~#1}

% Metadatenangaben
\author{\thesisauthor}
\title{\thesistitle}

\documentclass[\mainsize, a4paper, fleqn, xcolor=dvipsnames]{scrartcl}

% PDF-Inhalt such- und kopierbar machen
\usepackage[resetfonts]{cmap}

% Anpassungen für deutsche Sprache
\usepackage[utf8]{inputenc}
\usepackage[T1]{fontenc}
\usepackage[ngerman]{babel}

% Mehr Farben zulassen
\usepackage{xcolor}

% Grafiken
\usepackage{graphicx}
\graphicspath{{images/}}
\usepackage{float}

% Textschriftart ändern
\usepackage{lmodern}

% Verbessertes Fontrendering
\usepackage[final]{microtype}

% Seitenränder festlegen
\usepackage[a4paper,left=4cm,right=2cm,top=3.8cm,bottom=2cm]{geometry}

% 1,5 Zeilenabstand konfigurieren
\usepackage[onehalfspacing]{setspace}

% Zeileneinschub mit Abstand ersetzen
\usepackage{parskip}

% Fußnoten konfigurieren
\usepackage[perpage, flushmargin, hang]{footmisc}

% Inhaltsverzeichnis Aussehen konfigurieren und ins Inhaltsverzeichnis mit aufnehmen
\usepackage{tocbibind}
\usepackage{tocloft}
\renewcommand{\cftsecleader}{\cftdotfill{\cftdotsep}}
\setlength{\cftsubsecindent}{\idxsubsecindent}
\setlength{\cftsubsubsecindent}{\idxsubsubsecindent}

% Schönere Seitenangaben
\usepackage{fancyhdr}
\fancyhf{}
\fancyhead[EL,OR]{\thepage}
\fancyhead[ER]{\chaptername~\thechapter}
\fancyhead[LO]{\nouppercase{\leftmark}}

% Namensnennungen kursiv schreiben
\usepackage{xpatch}
\xpatchcmd{\citeauthor}{\begingroup}{\begingroup\em}{}{}

% Mathematik-Pakete
\usepackage{amsmath,amsthm,amssymb}
\setlength{\mathindent}{1cm}

% Automatisches Setzen von Anführungszeichen (\enquote, \foreignquote, \blockquote)
\usepackage[autostyle=true,german=quotes]{csquotes}

% Einbinden von PDFs
\usepackage{pdfpages}

% Beschriftung von Bildern
\usepackage[
	format=plain,
	margin=0.5cm,
	labelformat=simple,
	labelfont={bf},
	font=footnotesize
]{caption}

% Erlaubt die Erstellung von zusammengesetzten Grafiken
\usepackage{subcaption}

% Leere Anfangs- und Endzeilen vermeiden
\usepackage[defaultlines=4,all]{nowidow}

% Bessere Aufzählungen
\usepackage{paralist}

% Bessere Tabellen
\usepackage{booktabs}

% Euro-Zeichen
\usepackage[official]{eurosym}

% Verlinkungen und PDF Daten initialisieren
\usepackage{hyperref}
\hypersetup{
	pdftitle = {\thesistitle},
	pdfauthor = {\thesisauthor},
	pdfcreator = {pdflatex},
	pdfkeywords={\thesistype, \thesistitle},
	pdfsubject={\thesistype},
	pdfproducer = {LaTeX with hyperref},
	colorlinks = false,
	linktoc = all,
	linkbordercolor = {0 0 0.5}
}

\begin{document}

	%Namen der autoref generierten Verweise ändern
	\renewcommand{\sectionautorefname}{Kapitel}
	\renewcommand{\subsectionautorefname}{Abschnitt}
	\renewcommand{\subsubsectionautorefname}{Unterabschnitt}
	
	%Seitenränder abschalten
	\newgeometry{left=2cm,right=2cm,top=2cm,bottom=2cm}
	
	%Deckblatt einbinden
	%Seitenränder abschalten
\newgeometry{left=\titlemargin,right=\titlemargin,top=\titlemargin,bottom=\titlemargin}

\begin{titlepage}

   \singlespacing

   %Inhalt zentrieren
   \begin{center} \large 
    
    \universityname
    \vspace*{0.5cm}
    
    \facultyname
    \vspace*{1.5cm}
    
	\includegraphics[scale=0.45]{titlepage/\logofilename}
    \vspace*{2cm} 

	\thesistype \ zum Thema
	\vspace*{0.5cm}
	
    {\huge\textbf{\thesistitle}}
    \vspace*{1.5cm}
    
    
  	Zur Erlangung des Grades \studygoal
  	\vspace*{1.5cm}

    \textbf{\thesisauthor}
    \vspace*{0.5cm}
    
    Studiengang: \studytype \\
    Matrikelnummer: \studentnumber \\
    Fachsemester: \semesternumber \\
    E-Mail: \mail \\
    \vspace*{2.0cm}

    Abgabe am: \submissiondate \\
    (\semester)
    \vspace*{1.5cm}

    Betreuung durch: \supervisor
  \end{center}
  
\end{titlepage}

%Seitenränder restaurieren
\restoregeometry
	
	%Seitenränder restaurieren
	\restoregeometry
	
	%Leere Seite einfügen
	\newpage
	\thispagestyle{empty}
	\mbox{}
	\newpage
	
	%Seitenangaben aktivieren
	\pagestyle{fancy}
	
	%Vorverzeichnisse einbinden
	\begingroup

% Header auch für Seiten ohne Kopfzeile einfügen
\fancypagestyle{plain}

% Zeilenabstand ändern
\singlespacing

% Inhaltsverzeichnis einfügen
\tableofcontents
\newpage

% Abbildungsverzeichnis einfügen
\listoffigures
\newpage

% Tabellenverzeichnis einfügen
\listoftables
\newpage

% Quellcodeverzeichnis
\renewcommand{\lstlistoflistings}{\begingroup
\tocfile{\lstlistingname}{lol}
\endgroup}
\renewcommand{\lstlistingname}{Quellcodeverzeichnis}
\lstlistoflistings
\newpage

% Abkürzungsverzeichnis einfügen
\input{preindex/abbreviations}
\newpage

% Symbolverzeichnis erstellen
\section*{Symbolverzeichnis}
\markboth{Symbolverzeichnis}{Symbolverzeichnis}
\phantomsection
\addcontentsline{toc}{section}{\protect Symbolverzeichnis}

\begin{acronym}
	\itemsep-20pt
	% Hier Abkürzungen eintragen mit \acro{Kürzel}[Kurzform]{Langform} bzw. \acroplural{Kürzel}[Kurzform]{Langform}
	% Wird die Pluralform benötigt, so muss unter dem selben Kürzel auch die Singularform hinterlegt sein!
	% Im Text kann dann \ac{Kürzel} bzw. \acp{Kürzel} verwendet werden.
	% =============================================================================================================
	\acro{a}[\ensuremath{A}]{{\acrounit{\meter^2}Oberfläche}}
\end{acronym}
\newpage

% Formelverzeichnis erstellen
\newlistof{equations}{equ}{Formelverzeichnis}
\renewcommand{\listofequations}{\begingroup
\tocfile{Formelverzeichnis}{equ}
\endgroup}
\listofequations
\newpage

\endgroup

	%Testtext
	\section{Beispiele}\label{sec:beispiel}
	
	Lorenawdawdwadz awod awd aiwd awdu awoud awduioawud oiawudoawuoduawoduawoidiuawduawodu awd awd awudwad uawdawiduawdawd adwaddwda\footnote{Testfußnote}
	
	\subsection{Bilder/Abbildungen}\label{subsec:abbildungen}
	
	Einzelnes Bild
	
	\begin{figure}[H]
		\centering
		\begin{minipage}{0.6\textwidth}
			\centering
			\fbox{\includegraphics[width=\textwidth]{logo}}
			\source{Hier cite einfügen}
		\end{minipage}
		\caption{Logo der Hochschule München}
  		\label{fig:logo}
	\end{figure}
	
	Subfigures siehe \autorefpage{fig:coffee}
	
	\begin{figure}[H]
  		\centering
  		\begin{subfigure}[b]{0.4\textwidth}
    		\includegraphics[width=\textwidth]{logo}
    		\caption{A logo.}
  		\end{subfigure}
  		\begin{subfigure}[b]{0.4\textwidth}
    		\includegraphics[width=\textwidth]{logo}
    		\caption{More logo.}
  		\end{subfigure}
  		\caption{The same cup of coffee. Two times.}
  		\label{fig:coffee}
	\end{figure}
	
	Unabhängige Bilder (aber nebeneinander)
	
	\begin{figure}[H]
		\centering
		\begin{minipage}{.5\textwidth}
  			\centering
  			\includegraphics[width=\textwidth]{logo}
  			\caption{A figure}
  			\label{fig:test1}
		\end{minipage}%
		\begin{minipage}{.5\textwidth}
  			\centering
  			\includegraphics[width=\textwidth]{logo}
  			\caption{figure}
  			\label{fig:test2}
		\end{minipage}
	\end{figure}
	
	\subsection{Aufzählungen}\label{subsec:lists}
	
	Aufzählungen ohne voranstehende Beschriftung
	
	\begin{description}
		\item{Item 1}
		\item{Item 2}
		\item{Item 3}
	\end{description}
	
	Ungeordnete Aufzählungen
	
	\begin{itemize}
		\item{Item 1}
		\item{Item 2}
		\item{Item 3}
	\end{itemize}
	
	Geordnete Aufzählungen 
	
	\begin{enumerate} %[a/1/I/A] optional
		\item{Item 1}
		\begin{enumerate}
			\item Auch verschachtelt möglich
			\item Auch verschachtelt möglich
		\end{enumerate}
		\item{Item 2}
		\item{Item 3}
	\end{enumerate}
	
	Weitere Umgebungen sind compactenum und compactitem
	
	\subsection{Tabellen}\label{subsec:tables}
	
	
	
\end{document}