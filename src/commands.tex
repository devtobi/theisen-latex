% Befehl um Textbox für den Druck zu generieren
\newcommand{\calibrationbox}[2]{% parameters: #1=width, #2=height
	
	\setlength{\unitlength}{1.0mm}%
	\begin{picture}(#1,#2)%
	\linethickness{0.05mm}%
	\put(0,0){\dashbox{0.2}(#1,#2)%
	{\parbox{#1mm}{%
	\centering\footnotesize
	\textbf{Kontrollbox}\\ 
	Breite $ = #1 \textrm{mm}$\\
	Höhe $ = #2 \textrm{mm}$
	}}}\end{picture}
}

% Eigener Befehl für autoref + Seitenangabe
\newcommand{\autorefpage}[1]{\autoref{#1} (siehe Seite \pageref{#1})}

% Eigener Befehl für Sekundärzitat
\newcommand{\secondarycite}[4]{
	(\cite[#3]{#4}
	zit. n. \cite[#1]{#2})
}

% Befehle für Quellenangaben
\newcommand*{\sourceright}[1]{\par\raggedleft\footnotesize \textit{Quelle}:~#1}
\newcommand*{\sourceplaceholder}{\par\raggedleft\footnotesize\phantom{\textit{Quelle}}}
\newcommand*{\sourceafterright}[1]{\par\raggedleft\footnotesize \textit{Nach Quelle}:~#1}
\newcommand*{\sourcecenter}[1]{\par\centering\footnotesize \textit{Quelle}:~#1}
\newcommand*{\sourceaftercenter}[1]{\par\centering\footnotesize \textit{Nach Quelle}:~#1}

% Befehle für das Inkludieren von Source Files
\newcommand{\includecode}[3]{\lstinputlisting[caption=#2, language=#1,label=lst:#3]{code/#3}}
\newcommand{\includewebcode}[2]{\lstinputlisting[caption={#1}, style=htmlcssjs_appendix,label=lst:#2]{code/#2}}

% Befehle für das Hinzufügen eines Eintrages in das Formelverzeichnis
\newcommand{\equationentry}[1]{%
\addcontentsline{equ}{equations}{\protect\numberline{\theequation}#1}\par}

% Befehl für zusätzliche Spalte mit Einheiten in Symbolverzeichnis
\newcommand{\acrounit}[1]{
  \acroextra{
    \hspace{15mm}\makebox[20mm][l]{\si[per-mode=fraction,fraction-function=\sfrac]{#1}}
  }
}

% Befehl für einfachere Link-Fußnoten
\newcommand{\hyperfootnote}[1][]{\def\ArgI{{#1}}\hyperfootnoteRelay}
  % relay to new command to make extra optional command possible
\newcommand\hyperfootnoteRelay[2][]{\href{#1#2}{\ArgI}\footnote{\href{#1#2}{#2}}}
  % the first optional argument is now in \ArgI, the second is in #1
  
% Befehl für stacked-Balken-Diagramme (bzw. Umschalten zu neuem Balken in Gruppe)
\newcommand{\resetstackedplots}{
	\makeatletter
	\pgfplots@stacked@isfirstplottrue
	\makeatother
	\addplot [forget plot,draw=none] coordinates{(1,0) (2,0) (3,0)};
}