% Befehl um Textbox für den Druck zu generieren
\newcommand{\calibrationbox}[2]{% parameters: #1=width, #2=height
	
	\setlength{\unitlength}{1.0mm}%
	\begin{picture}(#1,#2)%
	\linethickness{0.05mm}%
	\put(0,0){\dashbox{0.2}(#1,#2)%
	{\parbox{#1mm}{%
	\centering\footnotesize
	\textbf{Kontrollbox}\\ 
	Breite $ = #1 \textrm{mm}$\\
	Höhe $ = #2 \textrm{mm}$
	}}}\end{picture}
}

% Eigener Befehl für autoref + Seitenangabe
\newcommand{\autorefpage}[1]{\autoref{#1} (S. \pageref{#1})}

% Befehl für Quellenangaben direkt unter dem Bild (rechts)
\newcommand*{\sourceright}[1]{\par\raggedleft\footnotesize \textit{Quelle}:~#1}

% Befehl für Quellenangaben direkt unter dem Bild (zentriert)
\newcommand*{\sourcecenter}[1]{\par\centering\footnotesize \textit{Quelle}:~#1}

% Befehle für das inkludieren von Source Files
\newcommand{\includecode}[3]{\lstinputlisting[caption=#2, language=#1,label=lst:#3]{code/#3}}

\newcommand{\includewebcode}[2]{\lstinputlisting[caption=#1, style=htmlcssjs,label=lst:#2]{code/#2}}

% Befehle für das Hinzufügen eines Eintrages in das Formelverzeichnis
\newcommand{\equationentry}[1]{%
\addcontentsline{equ}{equations}{\protect\numberline{\theequation}#1}\par}

% Befehl für zusätzliche Spalte mit Einheiten in Symbolverzeichnis
\newcommand{\acrounit}[1]{
  \acroextra{
    \hspace{15mm}\makebox[20mm][l]{\si[per-mode=fraction,fraction-function=\sfrac]{#1}}
  }
}